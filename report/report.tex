\documentclass[russian]{vegareport}

\title{Влияние макроэкономических и социальных показателей на уровень счастья}
\author{Сулейманов Асхаб, Заостровский Всеволод, Черепахин Иван}
\date{}
\usepackage[english,russian]{babel}

\begin{document}
    \maketitle

    \chapter{Введение}
        \section{Слова и философия}
        В чём смысл существования государства? Этим вопросом задавались тысячи мыслителей на протяжении всей истории человечества. Более того, очень многие из них были твёрдо убеждены в том, что именно им удалось услышать голос Истины, хотя, конечно, сценарий при котором одна и та же Истина, вмещающая в себя сущность госудерства, одновременно описывается миллионами эпитетов, аллегорий и фразеологизмов, колеблющихся от "самого холодного из всех холодных чудовищ" до "намордника для усмирения плотоядного животного, называющегося человеком",  кажется, по меньшей мере, весьма и весьма маловероятным. Мы, конечно, не ставим задачи углубиться в тысячелетнюю историю этой дискуссии, и отвечаем на этот вопрос до наивности безхитростно: смысл государства, равно как и всякого изобретения человечества, состоит в привнесении счастья в печальное настоящее нашего рода. 
        \\
        Однако, далеко не все творения покорно следуют замыслам своих создателей, потому интересно разобраться в том, каких побед человечеству удалось достичь на этом поприще и в каком направлении следует двигаться дальше. 
        \\
        Конечно, мы далеко не первые люди, которые задались подобными вопросами. Попытки отыскания альтернативных путей для оценки благополучия государств положили начало сравнительно новому направлению экономической теории --- экономике счастья. На данный момент написано весьма впечатляющее число работ, посвященных данной тематике, имеются замечательные результаты, ведётся статистика некоторых показателей.    
        \\
        К примеру, сайт \href{https://worldhappiness.report/}{World Happiness Report}
        встречает своих читателей словами, прекрасно, на наш взгляд, описывающими парадигму этого направления: "our success as countries should be judged by the happiness of our people". В рамках работы, именно \href{https://www.kaggle.com/datasets/mathurinache/world-happiness-report-20152021?select=2016.csv}{эти} данные мы будем использовать в качестве меры счастья. Таким образом, мы имеем сущность, которую без потери здравого смысла, можно положить мерой счастья. Конечно, число объектов подобного рода ограничено лишь фантазией исследователей и может сильно варьироваться от статьи к статье. В большей части исследований, ссылки на которые мы дадим, подход к измерению счастья совершенно иной. И всё же мы считаем, что эти результаты являются серьёзным основанием для построения моделей и интерпретации переменных. 
        \\
        Следующий вопрос, ответ на который необходимо сформулировать, такой: "как характеризовать государство и направление в котором оно движется?" Многообразие путей, которые можно избрать для исследования этого аспекта пугающе велико. Есть существенные основания полагать, что счастье имеет социокультурный генезис, поэтому возникает искушение углубиться в этот вопрос в рамках психологической или филосойской парадигмы. Тем не менее, сушественным недостатком этого подхода является сложность или невозможность формализации и дальнейшего количественного исследования результатов. Мы будем рассматривать исключительно объективные и измеримые показатели, по которым принято судить о развитости экономики государства, а также те, что могут существенно влиять на самоощущение его граждан.

        \section{Данные}
        Изначально мы рассматривали следующие объясняющие переменные \footnote{Исчерпывающее техническое описание данных и они сами могут быть найдены по соответствующим ссылкам}:
        \begin{enumerate}
            \item \href{https://data.worldbank.org/indicator/NY.GDP.PCAP.PP.CD}{ВВП по ППС на душу населения.}
            \item \href{https://data.worldbank.org/indicator/GB.XPD.RSDV.GD.ZS?view=chart}{Уровень безработицы.}
            \item \href{https://data.worldbank.org/indicator/IC.TAX.TOTL.CP.ZS?view=chart}{Налоговая нагрузка физических лиц.}
            \item \href{https://data.worldbank.org/indicator/FP.CPI.TOTL.ZG?view=chart}{Инфляция.}
            \item \href{https://data.worldbank.org/indicator/SH.XPD.CHEX.GD.ZS}{Государственные расходы на медицину.}
            \item \href{https://data.worldbank.org/indicator/SE.XPD.TOTL.GD.ZS?view=chart }{Государственные расходы на образование.}
            \item \href{https://data.worldbank.org/indicator/MS.MIL.XPND.GD.ZS}{Государственные военные расходы.}
            \item \href{https://data.worldbank.org/indicator/NY.GNS.ICTR.ZS?view=chart}{Накопления физических лиц.}
            \item \href{https://data.worldbank.org/indicator/IP.PAT.RESD?view=chart}{Число зарегестрированных патентов по отношению к населению государства.}
            \item \href{https://data.worldbank.org/indicator/SP.POP.DPND}{Демографическая нагрузка.}
            \item \href{https://data.worldbank.org/indicator/VC.IHR.PSRC.P5}{Число убийств на тысячу человек.}
        \end{enumerate}
        Мы собрали данные по этим показателям за 2015-2019 годы и интерпертировали их как панельные. В дальнейшем мы сформировали на их основе несколько моделей, речь о которых пойдёт в следующей главе.

    \chapter{Модели и результаты}
    \section{Обзор существующих исследований}
    Как отмечалось во введении, экономика счастья --- это весьма популярное направление исследований. Классиком этой области знаний считается Richard Easterlin, обративший внимание в статье \ref{Easterlin} на связь доходов и счастья людей. Долгое время исследователь продвигал идею того, что уровень счастья людей не зависит от их доходов. 
    \\
    Однако, в этом вопросе всё не так однозначно. В 2010 году лауреаты премии Нобеля по экономике, Daniel Kahneman1 и Angus Deaton, опубликовали исследование \ref{KahnemanDeaton}, в котором, на примере США, показали, что счастье растёт вместе с доходами до определённого уровня, после которого рост, по меньшеё мере, очень существенно замедляется. Этот вопрос на самом деле невероятно важен и очень неочевиден, потому нам кажется целесообразным включение ВВП в той или иной форме в любые модели предсказания счастья. Врочем, похоже, что это вполне соответствует общепринятому подходу к моделированию счастья: например, в \ref{DiTella} отмечалась важность роли ВВП. Этот показатель является регрессором во всех трёх рассматриваемых моделях. Причём, регрессию мы будем строить для логарифма подушевого ВВП по ППС, чтобы учесть эффект, описанный в начале этого абзаца.
    \\
    Уровень безработицы также является одним из наиболее классических индикаторов положения дел в государстве. К тому же, есть серьёзные основания полагать, что он должен влиять на чувство удовлетворённости человека. Вопрос связи счастья и безработицы представляет самостоятельный интерес, поэтому существует немало статей, исследующих это, например, в статье \ref{Winkelmann} было показано, что высокий уровень безработицы негативно влияет на счастье. В статье \ref{Clarkunemp} последний эффект уже называется "общепринятым", сама статья посвящена исследованию некоторых деталей описываемого явления, кроме того, в ней можно найти ссылки на ещё большее число работ, всесторонне исследующих вопрос связи счастья и безработицы. 
    \\
    Что касается налоговой нагрузки, то здесь она, с одной стороны, является индикатором активности государственной социальной политики, а с другой, на самом деле может влиять на уровень счастья. Например, в статье \ref{tax} рассмотрена ситуация, в которой доказано влияние подоходного налога на уровень счастья. 
    \\
    С инфляцией всё гораздо менее очевидно. С одной стороны, не вызывает сомнений негативное отношение людей к инфляции (причины этого феномена исследуются,например, в статье \ref{Inflation}). С другой стороны, \href{http://www.centralbanknews.info/p/inflation-targets.html}{центральные банки огромного количества стран осуществляют политику таргетирования инфляции}, так что её величина является следствием положения дел в экономике, а не причиной. Счастье же населения, в данном контексте, зависит лишь от того, насколько грамотны решения регулятора. Тем не менее, это очень значимый экономический показатель, поэтому мы включаем его во все три наши модели.
    \\
    Следующая категория объясняющих переменных связана со структурой государственного бюджета в двух наиболее значимых социальных сферах: в медицине (\ref{medicine}) и в образовании (\ref{Education}). 
    \\
    Последняя категория переменных была внесена в рассмотрение в попытке описания как можно более разнообразных черт экономического и социального положения государства. Это преследует цель снижения эндогенности модели. Кроме того, мы рассматриваем решение о рассмотрении этих переменных как попытку обнаружения новых показателей, значимо влияющих на уровень счастья. 

    \section{Модель с "классическими" объясняющими переменными}
    \section{Модель со "всеми" переменными}
    \chapter{Выводы}

    \begin{thebibliography}{100}
            \bibitem{[1]} \label{Easterlin}
            Easterlin R. A.
            Does Economic Growth Improve the Human Lot? Some Empirical Evidence.
            --- \textit{Nations and Households in Economic Growth}, 
            89-125, 1974.

            \bibitem{} \label{KahnemanDeaton}
            Kahneman D., Deaton A. 
            High income improves evaluation of life but not emotional well-being. 
            --- \textit{Proc Natl Acad Sci U S A},
            \textbf{38}, 2010.

            \bibitem{} \label{DiTella}
            Di Tella R., MacCulloch R. J., Oswald A. J.
            The macroeconomics of happiness. 
           --- \textit{Review of Economics and Statistics}, 
            \textbf{85}, 1999. 

            \bibitem{} \label{Easterlin2}
            Easterlin R. A.
            Will Raising the Incomes of All Increase the Happiness of All?
            --- \textit{Journal of Economic Behaviour and Organization},
            \textbf{27}, 35-48, 1999. 

            \bibitem{} \label{Clark}
            Clark A. E., Oswald A. J.
            Unhappiness and Unemployment.
            --- \textit{Economic Journal}, 
            \textbf{104}, 648-659, 1994. 

            \bibitem{} \label{Winkelmann}
            Winkelmann L, Winkelmann R.
            Why are the unemployed so unhappy?
            --- \textit{Economica}, 
            \textbf{65}, 1-15, 1998. 

            \bibitem{} \label{tax}
            Hutchinson T., Ahmed I., Buryi P.
            Impact of income tax on happiness: evidence from the United States. 
            --- \textit{Applied Economics Letters}, 1-3.
            1-3, 2016.

            \bibitem{} \label{Clarkunemp}
            Clark A. E.
            A Note on Unhappiness and Unemployment Duration.
            --- \textit{IZA Discussion Paper No. 2406}, 
            2006. 
            
            \bibitem{} \label{Inflation}
            Shiller J. R.
            Why Do People Dislike Inflation?
            --- \textit{NBER Working Paper}, 
            \textbf{5539}, 1996. 

            \bibitem{} \label{medicine}
            Dfarhud D., Maryam M., Khanahmadi M. 
            Happiness & Health: The Biological Factors- Systematic Review Article.
            --- \textit{Iranian Journal of Public Health}, 
            \textbf{43}, 1468-1477, 2014. 

            \bibitem{} \label{Education}
            Satoshi Araki.
            Does Education Make People Happy? Spotlighting the Overlooked Societal Condition.
            --- \textit{Journal of Happiness Studies}, 
            \textbf{23}, 587-629, 2021. 
        \end{thebibliography}


\end{document}